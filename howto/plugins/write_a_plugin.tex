\documentclass[11pt,a4paper,english,oneside,parskip=false]{scrartcl} %parskip=half
\usepackage[T1]{fontenc}
\usepackage[utf8]{inputenc}
\usepackage[english]{babel}

\usepackage{amsmath}
\usepackage{amsfonts}
\usepackage{amssymb}

%\usepackage[a4paper, top=24mm, bottom=24mm, left=24mm, right=24mm]{geometry}

%\usepackage{mathpazo}
\usepackage{helvet}

\usepackage{hyphenat}
\usepackage{url}
\usepackage{hyperref}
\hypersetup{%
	final=true,%
	colorlinks=true,%
	linkcolor=black,%
	citecolor=black,%
	urlcolor=black,%
}
\usepackage[babel=true]{csquotes}
\usepackage{graphicx}

%\usepackage{todonotes}

\linespread{1.05}
\renewcommand{\familydefault}{\sfdefault}
\fontfamily{phv}\selectfont
%\setkomafont{disposition}{\rmfamily}
\clubpenalty = 20000
\widowpenalty = 20000

\addto\extrasenglish{%
	\renewcommand{\sectionautorefname}{Section}
	\renewcommand{\subsectionautorefname}{Section}
	\renewcommand{\subsubsectionautorefname}{Section}
}

\usepackage{enumitem}
\newlist{compactitemize}{itemize}{3}
\setlist[compactitemize]{nosep,label={\raisebox{.1em}{\small\textbullet}}}
\newlist{compactenumerate}{enumerate}{3}
\setlist[compactenumerate]{nosep,label=\arabic*.}


\begin{document}

\title{Implementing a Plug\hyp{}In Service for OceanTEA}
%\author{Arne Johanson}
\date{}

\maketitle

Custom functionality can be included into the user interface of OceanTEA by implementing a plug\hyp{}in (micro-)service. 
After registering such a service with the OceanTEA API gateway, its functionality can be accessed via a new tab in OceanTEA. 

An example of such a plug\hyp{}in service is the OceanTEA Coral Analysis service,\footnote{\url{https://github.com/a-johanson/oceantea-coral-analysis}} which offers an interactive illustration of research results to accompany a scientific paper.

\section{Implement Your Service}

You can use any technology to develop a plug\hyp{}in service for OceanTEA as long as it provides a RESTful API accessible via HTTP which adheres to the following rules: 
\begin{enumerate}
	\item All REST endpoints your service provides must be prefixed with a fixed \texttt{APIPrefix} such as \texttt{/coral\_analysis}:\\
	\texttt{<APIPrefix>/path/to/your/endpoint}\\
	In this way, the API gateway of OceanTEA knows which requests are supposed to be routed to your service. 
	\item Your service must provide an HTML fragment at the URL\\
	\texttt{<APIPrefix>/static/index.html}\\
	which renders your tab within OceanTEA. 
	This fragment is included into the body of the main view template of OceanTEA.\footnote{\url{https://github.com/a-johanson/oceantea-visualization-gateway/blob/master/views/layouts/main.handlebars}}
	\item Any static content that you need to include into your view must be served by your service at the URL:\\
	\texttt{<APIPrefix>/static/<pathToContent>}\\
	From within your \texttt{index.html} fragment, any static content is to be addressed using exactly the same \emph{relative} URL \texttt{<APIPrefix>/static/<pathToContent>} (this results in the requests being routed through the API gateway). 
	%\item TODO: Authentication/Authorization
\end{enumerate}


\section{Integrate Your Service With OceanTEA} \label{sec:git}

To make your service known to OceanTEA, you have to register it with the OceanTEA API gateway. 
This is done by adding an entry to the \texttt{plugins} array in \texttt{plugins.js}.\footnote{\url{https://github.com/a-johanson/oceantea-visualization-gateway/blob/master/plugins.js}} 
Such an entry needs to be structured as follows:\footnote{For a full working example, see:\\ \url{https://github.com/a-johanson/oceantea-visualization-gateway/blob/master/plugins.js.corals}} 
\begin{verbatim}
{
  tabName: "P. arborea Activity",
  guiURL: "/p_arborea_activity.html",
  apiPrefix: "/coral_analysis",
  serviceName: "oceantea-coral-analysis",
  serviceHost: "localhost",
  servicePort: "3340",
  templateParameters: {
    libD3js : true,
    libCanvasPlot: true
  }
}
\end{verbatim}

The field \texttt{tabName} is the text to be displayed on your plug\hyp{}in tab in the OceanTEA GUI. 
\texttt{guiURL} is the URL at which your tab is located, and \texttt{apiPrefix} is the API prefix discussed in the previous section. 
\texttt{serviceName} is a unique identifier of your service to be used, for example, with service registries and with Docker. 
\texttt{serviceHost} and \texttt{servicePort} describe how to connect to your service. 

\texttt{templateParameters} are additional parameters to be passed to the main view template of OceanTEA to, e.g., load predefined libraries. 
The following parameters are available:
\begin{compactitemize}
	\item \texttt{libD3js: true}
	\item \texttt{libCanvasPlot: true}
	\item \texttt{libFontAwesome: true}
	\item \texttt{libBootstrapJS: true}
	\item \texttt{libBootstrapSlider: true}
\end{compactitemize}
By default, jQuery\footnote{\url{https://jquery.com}} and the Bootstrap\footnote{\url{https://getbootstrap.com}} CSS are loaded. 

\end{document}
