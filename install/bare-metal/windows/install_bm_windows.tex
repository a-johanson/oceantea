\documentclass[11pt,a4paper,english,oneside,parskip=false]{scrartcl} %parskip=half
\usepackage[T1]{fontenc}
\usepackage[utf8]{inputenc}
\usepackage[english]{babel}

\usepackage{amsmath}
\usepackage{amsfonts}
\usepackage{amssymb}

%\usepackage[a4paper, top=24mm, bottom=24mm, left=24mm, right=24mm]{geometry}

\usepackage{mathpazo}

\usepackage{hyphenat}
\usepackage{url}
\usepackage{hyperref}
\hypersetup{%
	final=true,%
	colorlinks=true,%
	linkcolor=black,%
	citecolor=black,%
	urlcolor=black,%
}
\usepackage[babel=true]{csquotes}
\usepackage{graphicx}

\usepackage{todonotes}

\linespread{1.05}
\setkomafont{disposition}{\rmfamily}
%\clubpenalty = 20000
%\widowpenalty = 20000

\addto\extrasenglish{%
	\renewcommand{\sectionautorefname}{Section}
	\renewcommand{\subsectionautorefname}{Section}
	\renewcommand{\subsubsectionautorefname}{Section}
}


\begin{document}

\title{Bare\hyp{}Metal Installation of OceanTEA on Microsoft Windows}
\author{Arne Johanson}
\date{}

\maketitle

\section{Overview}

It is assumed that you run a 64 bit version of Windows 7 or later. 
Before installing OceanTEA itself, you will install software that is required to run OceanTEA: 
\begin{enumerate}
	\item Git (version control system; \autoref{sec:git})
	\item NodeJS (JavaScript runtime environment; \autoref{sec:node})
	\item Anaconda (Python 3 distribution; \autoref{sec:anaconda})
\end{enumerate}
Afterwards, you will install OceanTEA, as is described in \autoref{sec:oceantea}. 


\section{Installing Git} \label{sec:git}

\begin{enumerate}
	\item Download the latest version of Git for Windows from:\\ \url{https://git-scm.com/download/win}
	\item Run the installer and accept all default settings. Make sure that on the screen entitled \enquote{} ... is selected like in the following screenshot:
\end{enumerate}



\section{Installing NodeJS} \label{sec:node}

\begin{enumerate}
	\item Download the latest version of NodeJS for Windows from:\\
	\url{https://nodejs.org/en/download/current/}
	\item Run the setup program and install NodeJS with default settings.
\end{enumerate}


\section{Installing Anaconda} \label{sec:anaconda}

Download the latest version of Anaconda with Python 3 for Windows from:\\
\url{https://www.continuum.io/downloads#_windows}\\
Make sure to download the version that comes with Python 3 and \emph{not} with Python 2!


\section{Installing OceanTEA} \label{sec:oceantea}




\end{document}
